\documentclass[12pt]{beamer}
\setbeamertemplate{navigation symbols}{}
\usetheme{Copenhagen}
\usepackage{listings}
\usepackage{xcolor}
\usepackage{graphicx}
\usepackage{hyperref}
\graphicspath{ {imagenes/} }

\definecolor{codegreen}{rgb}{0,0.6,0}
\definecolor{codegray}{rgb}{0.5,0.5,0.5}
\definecolor{codepurple}{rgb}{0.58,0,0.82}
\definecolor{backcolour}{rgb}{0.95,0.95,0.92}

\lstdefinestyle{mystyle}{
    language=c++,
    backgroundcolor=\color{backcolour},   
    commentstyle=\color{codegreen},
    keywordstyle=\color{magenta},
    numberstyle=\tiny\color{codegray},
    stringstyle=\color{codepurple},
    basicstyle=\ttfamily\footnotesize,
    breakatwhitespace=false,         
    breaklines=true,                 
    captionpos=b,                    
    keepspaces=true,                 
    numbers=left,                    
    numbersep=5pt,                  
    showspaces=false,                
    showstringspaces=false,
    showtabs=false,                  
    tabsize=2
}

\lstset{style=mystyle}

\title{Algoritmos y Estructuras de Datos}
\subtitle{Introducción a la práctica}
\author{Tomás Peiretti}
\date{}

\begin{document}

\maketitle

\begin{frame}{Dinámica de las clases de práctica}
    \begin{itemize}
        \item Repaso rápido de la teoría al principio de la clase
        \item Presentación de soluciones (a veces)
        \item Principalmente, \alert{resolución de problemas por su parte}
    \end{itemize}
    \bigskip
    \begin{columns}
        \column{0.8\textwidth}¿Dudas? Preguntar en clase, enviar por campus (foro o chat) o enviar por mail
        \column{0.25\textwidth}\includegraphics[width=\textwidth]{ok-cat.jpeg}
    \end{columns}
\end{frame}

\begin{frame}{Listado de temas}
    \begin{itemize}
        \item Tipos de dato y operadores
        \item Estructuras de control
        \item Funciones
        \item Arreglos
        \item Strings
        \item Structs
        \item Complejidad
        \item Archivos
        \item Punteros
    \end{itemize}
\end{frame}

\begin{frame}[fragile]{Estructura de un programa básico}
\begin{lstlisting}[basicstyle=\scriptsize]
// directivas al preprocesador
#include <iostream>
using namespace std;

// Declaraciones globales y compartidas
int gravedad = -9.81;
#define PI 3.1415

// funcion main: punto de inicio del programa
int main(){
    // definicion e iniciaizacion de variables
    int n1 = 20;
    int n2;
    cin >> n2;

    // procesamiento de los datos
    n1 = PI * n1 + n2;

    // salida por pantalla de los resultados
    cout << n1 << endl;
    return 0;
}
\end{lstlisting}
\end{frame}

\end{document}