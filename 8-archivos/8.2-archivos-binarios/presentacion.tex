\documentclass[12pt]{beamer}
\setbeamertemplate{navigation symbols}{}
\usetheme{Copenhagen}
\usepackage{listings}
\usepackage{xcolor}
\usepackage{graphicx}
\usepackage{hyperref}
\usepackage{multicol}
\graphicspath{ {imagenes/} }

\definecolor{codegreen}{rgb}{0,0.6,0}
\definecolor{codegray}{rgb}{0.5,0.5,0.5}
\definecolor{codepurple}{rgb}{0.58,0,0.82}
\definecolor{backcolour}{rgb}{0.95,0.95,0.92}

\lstdefinestyle{mystyle}{
    language=c++,
    backgroundcolor=\color{backcolour},   
    commentstyle=\color{codegreen},
    keywordstyle=\color{magenta},
    numberstyle=\tiny\color{codegray},
    stringstyle=\color{codepurple},
    basicstyle=\ttfamily\footnotesize,
    breakatwhitespace=false,         
    breaklines=true,                 
    captionpos=b,                    
    keepspaces=true,                 
    numbers=left,                    
    numbersep=5pt,                  
    showspaces=false,                
    showstringspaces=false,
    showtabs=false,                  
    tabsize=2
}

\lstset{style=mystyle}

\title{Archivos}
\subtitle{Archivos binarios}
\author{Tomás Peiretti}
\date{}

\begin{document}

\maketitle

\begin{frame}{Archivos binarios}
    Puntos clave:
    \begin{itemize}
        \item C++ trabaja cada archivo como un \alert{flujo secuencial de bytes} (podemos pensarlo como un arreglo de bytes)
        \item Permiten almacenar estructuras (struct) completas
        \item Es posible utilizar acceso directo para leer/escribir
        \item Hay 3 tipos:
        \begin{itemize}
            \item De entrada: \alert{ifstream}
            \item De salida: \alert{ofstream}
            \item De entrada/salida: \alert{fstream}
        \end{itemize}
    \end{itemize}
\end{frame}

\begin{frame}{Archivos binarios: implementación}
    Al igual que los archivos de texto, para utilizar un archivo binario en nuestro programa debemos:
        \begin{enumerate}
            \item Incluir la librería fstream
            \item Declarar la variable que actuará como manejador de fichero
            \item Abrir el flujo de datos, vinculando la variable correspondiente con el fichero especificado.
            \item Comprobar que la apertura del fichero se realizó correctamente
            \item \alert{Realizar la transferencia de información}
            \item Finalmente, cerrar el flujo para liberar la variable manejador de su vinculación con el fichero.
        \end{enumerate}
\end{frame}

\begin{frame}[fragile]{Archivos binarios: implementación}
\begin{lstlisting}[basicstyle=\scriptsize]
#include <iostream>
#include <fstream>  //1
using namespace std;

int main() {
	ifstream archivo; // 2
	archivo.open("text.data", ios::binary); // 3
	
	if (archivo.is_open()) { // 4
		int x;
		archivo.read((char *) &x, sizeof(x)); // 5
		cout << "Valor en el archivo = " << x << endl;
	}
	else {
		cout << "Error al abrir el archivo" << endl;
	}
	
	archivo.close(); // 6
	
	return 0;
}
\end{lstlisting}
\end{frame}

\begin{frame}{Leer de un archivo binario}
    Para leer un archivo binario usaremos el método \alert{read} de la siguiente manera: \\
    \begin{center}
         .read((char *) \& \alert{variable}, sizeof(\alert{variable}))
    \end{center}
    Esto leerá un bloque de datos (bytes) del tamaño de la variable y lo transformará al tipo de dato de la variable
\end{frame}

\begin{frame}[fragile]{Leer de un archivo binario: ejemplo}
\begin{lstlisting}[basicstyle=\tiny]
struct Figura {
	int cantLados;
	int perimetro;
	double area;
};

int main() {
	ifstream archFiguras;
	ifstream archNumeros;
	archFiguras.open("figuras.data", ios::binary);
	archNumeros.open("numeros.bin", ios::binary);
	
	
	if (archFiguras.is_open() && archNumeros.is_open()) {
		Figura f;
		int num;
		archFiguras.read((char *) &f, sizeof(f));
		archNumeros.read((char *) &num, sizeof(num));
		cout << "Numero en el archivo = " << num << endl;
		cout << "Perimetro de la figura en archivo = " << f.perimetro << endl;
	}
	else {
		cout << "Error al abrir los archivos" << endl;
	}
	
	archFiguras.close();
	archNumeros.close();
	
	return 0;
}
\end{lstlisting}
\end{frame}

\begin{frame}[fragile]{Escribir en un archivo binario}
    Para escribir un archivo binario usaremos el método \alert{write} de la siguiente manera: \\
    \begin{center}
        .write((char *) \& \alert{variable}, sizeof(\alert{variable})) \\
    \end{center}
    Esto grabará un bloque de datos (bytes) del tamaño de la variable con los bytes correspondientes \\
\end{frame}

\begin{frame}[fragile]{Escribir en un archivo binario: ejemplo}
\begin{lstlisting}[basicstyle=\tiny]
struct Figura {
	int cantLados;
	int perimetro;
	double area;
};

int main(int argc, char *argv[]) {
	ofstream archFiguras;
	ofstream archNumeros;
	archFiguras.open("figuras.data", ios::binary);
	archNumeros.open("numeros.bin", ios::binary);
	
	if (archFiguras.is_open() && archNumeros.is_open()) {
		Figura f = {4, 20, 25};
		int num = 96;
		archFiguras.write((char *) &f, sizeof(f));
		archNumeros.write((char *) &num, sizeof(num));
	}
	else {
		cout << "Error al abrir los archivos" << endl;
	}
	
	archFiguras.close();
	archNumeros.close();
	
	return 0;
}
\end{lstlisting}
\end{frame}

\begin{frame}[fragile]{Decidir donde leer/escribir}
\begin{lstlisting}[basicstyle=\tiny]
struct Figura {
    int cantLados;
    int perimetro;
    double area;
};
// Actualizar el perimetro y area de la 4ta figura (que es un cuadrado)
// para que se corresponda a un cuadrado de lado = 10
int main() {
    fstream archFiguras;
    archFiguras.open("figuras.data", ios::binary | ios::in | ios::out);
    // suponiendo que se pudo abrir el archivo,
    // calculo la posicion de inicio de los bytes
    //     que corresponden a la 4ta figura
    int posFig4 = sizeof(Figura) * 3;
    Figura figura4;
    // muevo el puntero de lectura del archivo
    archFiguras.seekg(posFig4);
    // leo los datos de la figura persistida
    archFiguras.read((char *) &figura4, sizeof(figura4));
    // actualizo el perimetro y el area
    figura4.perimetro = 40;
    figura4.area = 100;
    // muevo el puntero de escritura del archivo
    archFiguras.seekp(posFig4);
    // guardo los cambios sobre el archivo
    archFiguras.write((char *) &figura4, sizeof(figura4));

    archFiguras.close();
    return 0;
}
\end{lstlisting}
\end{frame}

\begin{frame}[fragile]{Yapa: Leer un arreglo completo}
\begin{lstlisting}[basicstyle=\tiny]
struct Figura {
    int cantLados;
    int perimetro;
    double area;
};

int main() {
    ifstream archFiguras;
    archFiguras.open("figuras.data", ios::binary);
    
    Figura figuras[100];
    if (archFiguras.is_open()) {
        archFiguras.read((char *) &figuras, sizeof(figuras));
	}
	else {
		cout << "Error al abrir los archivos" << endl;
	}

    archFiguras.close();
    return 0;
}
\end{lstlisting}
\end{frame}

\begin{frame}{Desafío}
    \alert{Desafío:} desarrollar una función que permita encontrar una figura a partir de la cantidad de lados provista. \\ 
    \medskip
    La búsqueda debe realizarse sobre un archivo binario de figuras, el cual contiene las figuras ordenadas por su cantidad de lados. 
\end{frame}

\end{document}