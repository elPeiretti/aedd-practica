\documentclass[12pt]{beamer}
\setbeamertemplate{navigation symbols}{}
\usetheme{Copenhagen}
\usepackage{listings}
\usepackage{xcolor}
\usepackage{graphicx}
\usepackage{hyperref}
\usepackage{multicol}
\graphicspath{ {imagenes/} }

\definecolor{codegreen}{rgb}{0,0.6,0}
\definecolor{codegray}{rgb}{0.5,0.5,0.5}
\definecolor{codepurple}{rgb}{0.58,0,0.82}
\definecolor{backcolour}{rgb}{0.95,0.95,0.92}

\lstdefinestyle{mystyle}{
    language=c++,
    backgroundcolor=\color{backcolour},   
    commentstyle=\color{codegreen},
    keywordstyle=\color{magenta},
    numberstyle=\tiny\color{codegray},
    stringstyle=\color{codepurple},
    basicstyle=\ttfamily\footnotesize,
    breakatwhitespace=false,         
    breaklines=true,                 
    captionpos=b,                    
    keepspaces=true,                 
    numbers=left,                    
    numbersep=5pt,                  
    showspaces=false,                
    showstringspaces=false,
    showtabs=false,                  
    tabsize=2
}

\lstset{style=mystyle}

\title{Estructuras (struct)}
\subtitle{Conceptos}
\author{Tomás Peiretti}
\date{}

\begin{document}

\maketitle

\begin{frame}[fragile]{Estructuras}
    Una estructura es un conjunto de datos agrupados bajo un mismo nombre. Estos datos se conocen como \alert{miembros} y pueden tener diferentes tipos.

    \medskip

    \begin{columns}
        \column{0.5\textwidth}\includegraphics[width=\textwidth]{sintaxis_struct.png}
        \column{0.5\textwidth}\begin{lstlisting}
struct Alumno {
    int edad;
    char carrera;
    int notas[45];
    int tl_notas;
};

int main() {
    Alumno a1, a2;
}
\end{lstlisting}
    \end{columns}
\end{frame}


\begin{frame}[fragile]{Estructuras: acceso a los miembros}
    Los miembros de la estructura pueden ser accedidos de manera directa haciendo uso del \alert{'.'} como se muestra en el siguiente ejemplo:

    \medskip
\begin{lstlisting}[basicstyle=\tiny]
struct Alumno {
    int edad;
    char carrera;
    int notas[45];
    int tl_notas = 0;
} a1;

int main() {
    a1.edad = 19;
    a1.carrera = 'S';
    a1.notas[0] = 10;
    a1.notas[1] = 9;
    a1.tl_notas = 2;
    
    cout << "edad: " << a1.edad << endl; // imprime 19
    cout << "carrera: " << a1.carrera << endl; // imprime S
    cout << "nota en AEDD: " << a1.notas[0] << endl; // imprime 10 
}
\end{lstlisting}
\end{frame}

\begin{frame}[fragile]{Estructuras: anidaciones}
    Además, es posible utilizar estructuras dentro de otras estructuras:

    \medskip
\begin{lstlisting}[basicstyle=\tiny]
struct Punto {
    int x;
    int y;
};

struct Recta {
    Punto punto;
    double pendiente;
};

int main() {
    Punto p1 = {0, 10};
    Recta r1;
    r1.punto = p1;
    r1.pendiente = 2.5;

    cout << "La recta tiene pendiente = " << r1.pendiente << endl;
    cout << "y pasa por (" << r1.punto.x << "," << r1.punto.y << ")" << endl;
}
\end{lstlisting}
\end{frame}

\begin{frame}[fragile]{Estructuras: pasaje por parámetros}
    \centering Todas las estructuras por defecto \alert{pasan por copia}.

    \medskip
\begin{lstlisting}[basicstyle=\tiny]
struct Punto {
    int x;
    int y;
};
struct Recta {
    Punto punto;
    double pendiente;
};

void limpiarPunto1(Recta & r) {
    r.punto.x = 0;
    r.punto.y = 0;
}
void limpiarPunto2(Recta r) {
    r.punto.x=-1;
    r.punto.y=-1;
}

int main() {
    Recta r1 = {{5, 10}, 2.5};
    limpiarPunto2(r1);
    // como no pasa por referencia, imprime (5,10)
    cout << "(" << r1.punto.x << "," << r1.punto.y << ")" << endl;
    limpiarPunto1(r1);
    // como pasa por referencia, imprime (0,0)
    cout << "(" << r1.punto.x << "," << r1.punto.y << ")" << endl;
}
\end{lstlisting}
\end{frame}

\begin{frame}{Ejercicio}
    Supongamos el siguiente problema: \\
    \medskip
    Se deben leer los datos de diferentes animales e imprimir el nombre de los animales \alert{ordenados segun su altura}. \\
    \medskip
    Primero se ingresa un entero N, que indica la cantidad de animales. Luego siguen N lineas, cada una indicando el nombre, la altura (en cm) y el peso (en kg) de un animal
\end{frame}

\begin{frame}[fragile]{Ejercicio: solución}
\begin{lstlisting}[basicstyle=\tiny]
struct Animal {
    string nombre;
    double peso;
    double altura;
};

void ordenarPorAltura(Animal a[], int tl);

int main() {
    int n;
    Animal animales[10000];

    cin >> n;
    for (int i=0; i<n; i++)
        cin >> animales[i].nombre >> animales[i].peso >> animales[i].altura;

    ordenarPorAltura(animales, n);
    for (int i=0; i<n; i++)
       cout << animales[i].nombre << endl;
}

void ordenarPorAltura(Animal a[], int tl) {
    for (int i=0; i<tl; i++) {
        for(int j=i+1; j<tl; j++) {
            if (a[i].altura > a[j].altura)
                swap(a[i], a[j]);
        }
    }
}
\end{lstlisting}
\end{frame}

\begin{frame}{Ejercicios}
    \begin{itemize}
        \item \href{https://judge.beecrowd.com/es/problems/view/1258}{Beecrowd 1258}
        \item Ejercicios de la guía práctica de estructuras
    \end{itemize}
\end{frame}

\end{document}