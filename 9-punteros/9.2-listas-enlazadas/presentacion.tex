\documentclass[12pt]{beamer}
\setbeamertemplate{navigation symbols}{}
\usetheme{Copenhagen}
\usepackage{listings}
\usepackage{xcolor}
\usepackage{graphicx}
\usepackage{hyperref}
\usepackage{multicol}
\usepackage{multirow}
\graphicspath{ {imagenes/} }

\definecolor{codegreen}{rgb}{0,0.6,0}
\definecolor{codegray}{rgb}{0.5,0.5,0.5}
\definecolor{codepurple}{rgb}{0.58,0,0.82}
\definecolor{backcolour}{rgb}{0.95,0.95,0.92}

\lstdefinestyle{mystyle}{
    language=c++,
    backgroundcolor=\color{backcolour},   
    commentstyle=\color{codegreen},
    keywordstyle=\color{magenta},
    numberstyle=\tiny\color{codegray},
    stringstyle=\color{codepurple},
    basicstyle=\ttfamily\footnotesize,
    breakatwhitespace=false,         
    breaklines=true,                 
    captionpos=b,                    
    keepspaces=true,                 
    numbers=left,                    
    numbersep=5pt,                  
    showspaces=false,                
    showstringspaces=false,
    showtabs=false,                  
    tabsize=2
}

\lstset{style=mystyle}

\title{Punteros}
\subtitle{Listas Enlazadas}
\author{Tomás Peiretti}
\date{}

\begin{document}

\maketitle

\begin{frame}{Listas enlazadas}
    Una lista enlazada es una \alert{estructura de datos dinámica} que almacena los datos de forma secuencial, no contigua.
    \begin{itemize}
        \item Que sea dinámica significa que su tamaño físico puede variar durante tiempo de ejecución y que sus elementos se almacenan en diferentes porciones de la memoria.
    \end{itemize}
    \begin{table}[]
        \centering
        \begin{tabular}{|c|c|c|}
            \hline
            Operación & Arreglos & Listas enlazadas \\ \hline
            Buscar un elemento & O(N) & O(N) \\ \hline
            Acceder a un elemento & O(1) & O(N) \\ \hline
            Eliminar un elemento & O(N) & O(1) \\ \hline
            Agregar un elemento & O(N) & O(1) \\ \hline
        \end{tabular}
    \end{table}
    \includegraphics[width=\textwidth]{lista_enlazada.png}
\end{frame}

\begin{frame}[fragile]{Listas enlazadas: implementación}
   \includegraphics[width=\textwidth]{lista_enlazada.png}
\begin{lstlisting}[basicstyle=\tiny]
// cada elemento (nodo) contiene datos/informacion del elemento y
// la referencia (un puntero) al nodo siguiente.
// Ejemplo: Lista enlazada de enteros
struct Nodo {
    int num;
    Nodo * sig;
};
// Ejemplo: Lista enlazada de Personas
struct Persona {
    string nombre;
    int edad;
    int altura;
};

struct NodoP {
    Persona p;
    NodoP * sig;
};

int main() {
    // para manejar la lista enlazada,
    // utilizaremos un puntero cabecera (HEAD)
    // el cual apuntara al primer elemento de la lista
    Nodo * listaEnteros;
    NodoP * listaPersonas;
}
\end{lstlisting}
\end{frame}

\begin{frame}[fragile]{Listas enlazadas: creación y eliminación de nodos}
    Para crear un nuevo nodo/elemento, debemos reservar memoria. Para esto, utilizaremos la instrucción \alert{new}. \\
    \medskip
    \alert{new} permite reservar memoria para una cantidad especifica de bytes y retorna un puntero al objeto creado. En el caso de que no haya memoria, retorna un puntero NULL. \\
    \medskip
\begin{lstlisting}[basicstyle=\tiny]
struct Nodo {
    int num;
    Nodo * sig;
};

int main() {
    // creamos una lista enlazada con un unico elemento
    Nodo * lista = new Nodo;
    // chequeamos si se pudo crear el nodo
    if (lista != NULL) {
        lista->num = 100;
        lista->sig = NULL;
    } else {
        cout << "ERROR - NO HAY MEMORIA DISPONIBLE" << endl;
    }
    // Luego, al eliminar un elemento, debemos liberar la memoria
    delete lista;
    lista = NULL;
}
\end{lstlisting}
\end{frame}

\begin{frame}{Listas enlazadas: Operaciones}
    ¿Qué operaciones debemos saber realizar con las listas enlazadas para aprobar AEDD?
    \begin{columns}
        \column{0.6\textwidth}\begin{itemize}
            \item Eliminar/agregar un elemento
            \item Buscar un elemento
            \item Unir dos listas
            \item Invertir una lista
            \item Ordenar (SelectionSort, InsertionSort, etc)
        \end{itemize}
        \column{0.5\textwidth}\includegraphics[width=\textwidth]{linked_meme.jpg}
    \end{columns}
\end{frame}

\begin{frame}[fragile]{Operaciones: Creación de nodo}
\begin{lstlisting}[basicstyle=\tiny]
struct Nodo {
    int num;
    Nodo * sig = NULL;
};

bool agregarNodo(int num, Nodo * &lista) {
    Nodo * nuevo = new Nodo;
    if (nuevo == NULL) {
        return false;
    }
    
    nuevo->num = num;
    // si la lista esta vacia
    if (lista == NULL) {
        lista = nuevo;
        return true;
    }

    // para agregar un nodo, debemos ir hasta el ultimo
    Nodo * aux = lista;
    while (aux->sig != NULL) {
        aux = aux->sig;
    }
    // teniendo el ultimo nodo, agregamos el nuevo
    aux -> sig = nuevo;
    return true;
}
\end{lstlisting}
\end{frame}

\begin{frame}[fragile]{Operaciones: Creación de nodo 2}
\begin{lstlisting}[basicstyle=\tiny]
bool agregarNodo(int num, int pos, Nodo * &lista) {
    Nodo * nuevo = new Nodo;
    if (nuevo == NULL) {
        return false;
    }
    nuevo->num = num;
    // si se desea agregar como primer elemento:
    if (pos == 0) {
        nuevo->sig = lista;
        lista = nuevo;
        return true;
    }
    // para agregar un nodo en pos, debemos ir hasta la posicion previa
    Nodo * aux = lista;
    int i = 0;
    while (aux != NULL && i<pos-1) {
        aux = aux->sig;
        i++;
    }
    if (aux == NULL) {
        delete nuevo;
        return false;
    }
    
    nuevo->sig = aux->sig;
    aux->sig = nuevo;
    return true;
}
\end{lstlisting}
\end{frame}

\begin{frame}[fragile]{Operaciones: Buscar un elemento}
\begin{lstlisting}[basicstyle=\scriptsize]
struct Nodo {
    int num;
    Nodo * sig = NULL;
};

void buscarNum(int num, Nodo * lista) {
    int pos = 0;
    while(lista != NULL && lista->num != num) {
        lista = lista->sig;
        pos++;
    }
    
    if (lista == NULL) {
        cout << "No se ha encontrado a " << num << " en la lista"<<endl;
    } else {
        cout << num << " se encuentra en el nodo " << pos << endl;
    }
}
\end{lstlisting}
\end{frame}

\begin{frame}[fragile]{Operaciones: Eliminar un elemento}
\begin{lstlisting}[basicstyle=\tiny]
struct Nodo {
    int num;
    Nodo * sig = NULL;
};

void eliminar(int num, Nodo * &lista) {
    Nodo * aux = lista;
    Nodo * prev = NULL;
    while (aux != NULL && aux->num != num) {
        prev = aux;
        aux = aux->sig;
    }

    // nada que eliminar, no se encontro
    if (aux == NULL) {
        return;
    }
    
    // el nodo a eliminar es el primero
    if (prev == NULL) {
        lista = aux->sig;
    } else {
        prev->sig = aux->sig;
    }
    delete aux;
}
\end{lstlisting}
\end{frame}

\begin{frame}[fragile]{Operaciones: Invertir una lista}
\begin{lstlisting}[basicstyle=\scriptsize]
struct Nodo {
    int num;
    Nodo * sig = NULL;
};

void invertir(Nodo * &lista) {
    // lista vacia, nada que invertir
    if (lista == NULL) {
        return;
    }

    Nodo * prev = NULL;
    Nodo * actual = lista;

    while (actual != NULL) {
        Nodo * sig = actual->sig;
        actual->sig = prev;
        prev = actual;
        actual = sig;
    }
    lista = prev;
}
\end{lstlisting}
\end{frame}

\begin{frame}{Ejercicios}
    \begin{itemize}
        \item Guía de ejercicios de práctica de la cátedra
    \end{itemize}
\end{frame}

\end{document}