\documentclass[12pt]{beamer}
\setbeamertemplate{navigation symbols}{}
\usetheme{Copenhagen}
\usepackage{listings}
\usepackage{xcolor}
\usepackage{graphicx}
\usepackage{hyperref}
\graphicspath{ {imagenes/} }
\usepackage{multirow}
\usepackage{multicol}
\usepackage{colortbl}
\usepackage[T1]{fontenc}
\usepackage{microtype}
\DisableLigatures{}


\definecolor{codegreen}{rgb}{0,0.6,0}
\definecolor{codegray}{rgb}{0.5,0.5,0.5}
\definecolor{codepurple}{rgb}{0.58,0,0.82}
\definecolor{backcolour}{rgb}{0.95,0.95,0.92}

\lstdefinestyle{mystyle}{
    language=c++,
    backgroundcolor=\color{backcolour},   
    commentstyle=\color{codegreen},
    keywordstyle=\color{magenta},
    numberstyle=\tiny\color{codegray},
    stringstyle=\color{codepurple},
    basicstyle=\ttfamily\footnotesize,
    breakatwhitespace=false,         
    breaklines=true,                 
    captionpos=b,                    
    keepspaces=true,                 
    numbers=left,                    
    numbersep=5pt,                  
    showspaces=false,                
    showstringspaces=false,
    showtabs=false,                  
    tabsize=2
}

\lstset{style=mystyle}

\title{Introducción a C++}
\subtitle{Tipos de datos y Operadores}
\author{Tomás Peiretti}
\date{}

\begin{document}

\maketitle

\begin{frame}{Tipos de datos fundamentales}
    \begin{columns}
        \column{0.8\textwidth}Una variable es una porción de memoria que almacena un valor. Pero, ¿Qué \alert{tipos de valores} se pueden almacenar?
        \column{0.2\textwidth}\includegraphics[width=\textwidth]{thinking.png}
    \end{columns}

    \medskip

    Dentro de los tipos de datos fundamentales de c++ existen:
    \begin{itemize}
        \item Caracteres
        \item Números enteros
        \item Números de punto flotante
        \item Booleanos (valores de verdad)
    \end{itemize}
\end{frame}

\begin{frame}{Tipos de datos fundamentales}
    \begin{table}[]
        \tiny
        \centering
        \begin{tabular}{|l|l|l|l|}
            \hline
            \rowcolor[HTML]{E0E0E0} 
            \multicolumn{1}{|c|}{\cellcolor[HTML]{E0E0E0}\textbf{Grupo}} & \multicolumn{1}{c|}{\cellcolor[HTML]{E0E0E0}\textbf{Nombres de tipo}} & \multicolumn{1}{c|}{\cellcolor[HTML]{E0E0E0}\textbf{\begin{tabular}[c]{@{}c@{}}Precisión/Tamaño\end{tabular}}} & \multicolumn{1}{c|}{\cellcolor[HTML]{E0E0E0}\textbf{Rango de valores}} \\ \hline
            \rowcolor[HTML]{90DFFF} 
            \cellcolor[HTML]{FFFFFF} & \textbf{char} & \textbf{1 byte = 8 bits} & \textbf{0 a 255} \\ \cline{2-4} 
            \rowcolor[HTML]{FFFFFF} 
            \cellcolor[HTML]{FFFFFF} & \textbf{char16\_t} & 2 bytes = 16 bits & 0 a 65,535 \\ \cline{2-4} 
            \rowcolor[HTML]{FFFFFF} 
            \cellcolor[HTML]{FFFFFF} & \textbf{char32\_t} & 4 bytes = 32 bits & 0 a 4,294,697,295 \\ \cline{2-4} 
            \rowcolor[HTML]{FFFFFF} 
            \multirow{-4}{*}{\cellcolor[HTML]{FFFFFF}Caracteres} & \textbf{wchar\_t} & \begin{tabular}[c]{@{}l@{}}Puede almacenar \\cualquier\\ caracter soportado\end{tabular} &  \\ \hline
            \rowcolor[HTML]{FFFFFF} 
            \cellcolor[HTML]{FFFFFF} & \textbf{short} & 2 bytes = 16 bits & -32,768 a 32,767 \\ \cline{2-4} 
            \rowcolor[HTML]{90DFFF} 
            \cellcolor[HTML]{FFFFFF} & \textbf{int} & \textbf{4 bytes = 32 bits} & \textbf{-2,147,483,684 a 2,147,483,647} \\ \cline{2-4} 
            \rowcolor[HTML]{FFFFFF} 
            \cellcolor[HTML]{FFFFFF} & \textbf{long} & 4 bytes = 32 bits & -2,147,483,684 a 2,147,483,647 \\ \cline{2-4} 
            \rowcolor[HTML]{FFFFFF} 
            \multirow{-4}{*}{\cellcolor[HTML]{FFFFFF}\begin{tabular}[c]{@{}l@{}}Números\\ enteros\\ con signo\end{tabular}} & \textbf{long long} & \textbf{8 bytes = 64 bits} & \textbf{\begin{tabular}[c]{@{}l@{}}-9,223,372,036,854,775,808 a\\  9,223,372,036,854,775,807\end{tabular}} \\ \hline
            \rowcolor[HTML]{FFFFFF} 
            \cellcolor[HTML]{FFFFFF} & \textbf{unsigned short} & \cellcolor[HTML]{FFFFFF} & 0 a 65,535 \\ \cline{2-2} \cline{4-4} 
            \rowcolor[HTML]{FFFFFF} 
            \cellcolor[HTML]{FFFFFF} & \textbf{unsigned} & \cellcolor[HTML]{FFFFFF} & 0 a 4,294,967,295 \\ \cline{2-2} \cline{4-4} 
            \rowcolor[HTML]{FFFFFF} 
            \cellcolor[HTML]{FFFFFF} & \textbf{unsigned long} & \cellcolor[HTML]{FFFFFF} & 0 a 4,294,967,295 \\ \cline{2-2} \cline{4-4} 
            \rowcolor[HTML]{FFFFFF} 
            \multirow{-4}{*}{\cellcolor[HTML]{FFFFFF}\begin{tabular}[c]{@{}l@{}}Números\\ enteros\\ sin signo\end{tabular}} & \textbf{unsigned long long} & \multirow{-4}{*}{\cellcolor[HTML]{FFFFFF}\begin{tabular}[c]{@{}l@{}}Mismo tamaño que\\ los enteros con signo\end{tabular}} & 0 a 18,446,744,073,709,551,615 \\ \hline
            \rowcolor[HTML]{FFFFFF} 
            \cellcolor[HTML]{FFFFFF} & \cellcolor[HTML]{90DFFF}\textbf{float} & \cellcolor[HTML]{90DFFF}\textbf{4 bytes = 32 bits} & \cellcolor[HTML]{FFFFFF} \\ \cline{2-3}
            \rowcolor[HTML]{FFFFFF} 
            \cellcolor[HTML]{FFFFFF} & \cellcolor[HTML]{90DFFF}\textbf{double} & \cellcolor[HTML]{90DFFF}\textbf{8 bytes = 64 bits} & \cellcolor[HTML]{FFFFFF} \\ \cline{2-3}
            \rowcolor[HTML]{FFFFFF} 
            \multirow{-3}{*}{\cellcolor[HTML]{FFFFFF}\begin{tabular}[c]{@{}l@{}}Números de \\ punto flotante\end{tabular}} & \textbf{long double} & 8 bytes o mayor & \multirow{-3}{*}{\cellcolor[HTML]{FFFFFF}\href{https://en.cppreference.com/w/cpp/language/types}{Ver docs}} \\ \hline
            \rowcolor[HTML]{90DFFF} 
            \cellcolor[HTML]{FFFFFF}Boolean & \textbf{bool} & \textbf{True o false} & \textbf{true, false} \\ \hline
            \rowcolor[HTML]{FFFFFF} 
            Void & \textbf{void} & No almacena valor &  \\ \hline
        \end{tabular}
    \end{table}
    \centering\footnotesize\url{https://en.cppreference.com/w/cpp/language/types}
\end{frame}

\begin{frame}{Operadores}
    \begin{table}[]
        \centering
        \tiny
        \def\arraystretch{2}
        \begin{tabular}{lccl|l|c|c|}
            \cline{1-3} \cline{5-7}
            \multicolumn{1}{|c|}{\cellcolor[HTML]{E0E0E0}\textbf{Grupo}} & \multicolumn{1}{c|}{\cellcolor[HTML]{E0E0E0}\textbf{Operador}} & \multicolumn{1}{c|}{\cellcolor[HTML]{E0E0E0}\textbf{Ejemplo}} &  & \multicolumn{1}{c|}{\cellcolor[HTML]{E0E0E0}\textbf{Grupo}} & \cellcolor[HTML]{E0E0E0}\textbf{Operador} & \cellcolor[HTML]{E0E0E0}\textbf{Ejemplo} \\ \cline{1-3} \cline{5-7} 
            \multicolumn{1}{|l|}{} & \multicolumn{1}{c|}{\textbf{+}} & \multicolumn{1}{c|}{a + b} &  &  & \textbf{==} & x ==  b \\ \cline{2-3} \cline{6-7} 
            \multicolumn{1}{|l|}{} & \multicolumn{1}{c|}{\textbf{-}} & \multicolumn{1}{c|}{a - 5} &  &  & != & x != 'a' \\ \cline{2-3} \cline{6-7} 
            \multicolumn{1}{|l|}{} & \multicolumn{1}{c|}{\textbf{*}} & \multicolumn{1}{c|}{x * y} &  &  & \textbf{\textless{}} & x \textless z \\ \cline{2-3} \cline{6-7} 
            \multicolumn{1}{|l|}{} & \multicolumn{1}{c|}{\textbf{/}} & \multicolumn{1}{c|}{x / y} &  &  & \textgreater{} & x \textgreater 100 \\ \cline{2-3} \cline{6-7} 
            \multicolumn{1}{|l|}{\multirow{-5}{*}{Aritméticos}} & \multicolumn{1}{c|}{\%} & \multicolumn{1}{c|}{a \% 10} &  &  & \textless{}= & x \textless{}= y \\ \cline{1-3} \cline{6-7} 
            \multicolumn{1}{|l|}{} & \multicolumn{1}{c|}{\textbf{=}} & \multicolumn{1}{c|}{a = 1} &  & \multirow{-6}{*}{Relacionales} & \textgreater{}= & x \textgreater{}= 0 \\ \cline{2-3} \cline{5-7} 
            \multicolumn{1}{|l|}{} & \multicolumn{1}{c|}{\textbf{+=}} & \multicolumn{1}{c|}{a += 1} &  &  & ! & ! (x \textgreater{}= 0) \\ \cline{2-3} \cline{6-7} 
            \multicolumn{1}{|l|}{} & \multicolumn{1}{c|}{\textbf{-=}} & \multicolumn{1}{c|}{a -= 1} &  &  & \&\& & x \textgreater 0 \&\& x \textless{}= 100 \\ \cline{2-3} \cline{6-7} 
            \multicolumn{1}{|l|}{} & \multicolumn{1}{c|}{\textbf{*=}} & \multicolumn{1}{c|}{a *= 2} &  & \multirow{-3}{*}{Lógicos} & || & x \textless -10 || x \textgreater 10 \\ \cline{2-3} \cline{5-7} 
            \multicolumn{1}{|l|}{\multirow{-5}{*}{Asignación}} & \multicolumn{1}{c|}{/=} & \multicolumn{1}{c|}{a /= 10} &  &  & \textbf{++} & x++ o ++x \\ \cline{1-3} \cline{6-7} 
            & \multicolumn{1}{l}{} & \multicolumn{1}{l}{} &  &  & -- & x-- o --x \\ \cline{6-7} 
            & \multicolumn{1}{l}{} & \multicolumn{1}{l}{} &  & \multirow{-3}{*}{Otros} & sizeof & sizeof(x) \\ \cline{5-7} 
        \end{tabular}
    \end{table}
    \centering\footnotesize\url{https://cplusplus.com/doc/tutorial/operators}
\end{frame}

\begin{frame}[fragile]{Operador de "casteo" de tipo}
    Como sabemos, el compilador de C++ lleva a cabo conversiones implícitas. Por lo que, en algunos casos, necesitamos transformar un tipo de dato para poder manipularlo. 

    \medskip

    El operador de casteo permite convertir un valor que es de un determinado tipo de dato a otro tipo de dato. Hay dos maneras de utilizarlo:
\begin{lstlisting}[basicstyle=\tiny]
int main() {
    float x = 101.25;
    // casteando x a int para poder usar %
    cout << (int) x % 10 << endl;
    // otra manera de castear x a int
    cout << int(x) % 10 << endl;
    // ejemplo: castear el dividendo para que el resultado de la division sea float y no int
    cout << "1/2 = " << 1/2 << endl; // 1/2 = 0
    cout << "1/2 = " << float(1)/2 << endl; // 1/2 = 0.5
    return 0;
}
\end{lstlisting}
\end{frame}

\begin{frame}{Ejercicios}
    \begin{multicols}{2}
        \begin{itemize}
            \item \href{https://judge.beecrowd.com/es/problems/view/1001}{Beecrowd 1001}
            \item \href{https://judge.beecrowd.com/es/problems/view/1002}{Beecrowd 1002}
            \item \href{https://judge.beecrowd.com/es/problems/view/1003}{Beecrowd 1003}
            \item \href{https://judge.beecrowd.com/es/problems/view/1004}{Beecrowd 1004}
            \item \href{https://judge.beecrowd.com/es/problems/view/1005}{Beecrowd 1005}
            \item \href{https://judge.beecrowd.com/es/problems/view/1006}{Beecrowd 1006}
            \item \href{https://judge.beecrowd.com/es/problems/view/1007}{Beecrowd 1007}
            \item \href{https://judge.beecrowd.com/es/problems/view/1008}{Beecrowd 1008}
            \item \href{https://judge.beecrowd.com/es/problems/view/1009}{Beecrowd 1009}
            \item \href{https://judge.beecrowd.com/es/problems/view/1010}{Beecrowd 1010}
            \item \href{https://judge.beecrowd.com/es/problems/view/1011}{Beecrowd 1011}
            \item \href{https://judge.beecrowd.com/es/problems/view/1012}{Beecrowd 1012}
            \item \href{https://judge.beecrowd.com/es/problems/view/1013}{Beecrowd 1013}
            \item \href{https://judge.beecrowd.com/es/problems/view/1014}{Beecrowd 1014}
            \item \href{https://judge.beecrowd.com/es/problems/view/1015}{Beecrowd 1015}
            \item \href{https://judge.beecrowd.com/es/problems/view/1016}{Beecrowd 1016}
            \item \href{https://judge.beecrowd.com/es/problems/view/1017}{Beecrowd 1017}
            \item \href{https://judge.beecrowd.com/es/problems/view/1018}{Beecrowd 1018}
            \item \href{https://judge.beecrowd.com/es/problems/view/1019}{Beecrowd 1019}
            \item \href{https://judge.beecrowd.com/es/problems/view/1020}{Beecrowd 1020}
            \item \href{https://judge.beecrowd.com/es/problems/view/1021}{Beecrowd 1021}
        \end{itemize}
    \end{multicols}
\end{frame}

\end{document}