\documentclass[12pt]{beamer}
\setbeamertemplate{navigation symbols}{}
\usetheme{Copenhagen}
\usepackage{listings}
\usepackage{xcolor}
\usepackage{graphicx}
\usepackage{hyperref}
\usepackage{multicol}
\usepackage{microtype}
\usepackage[T1]{fontenc}
\DisableLigatures{}
\graphicspath{ {imagenes/} }

\definecolor{codegreen}{rgb}{0,0.6,0}
\definecolor{codegray}{rgb}{0.5,0.5,0.5}
\definecolor{codepurple}{rgb}{0.58,0,0.82}
\definecolor{backcolour}{rgb}{0.95,0.95,0.92}

\lstdefinestyle{mystyle}{
    language=c++,
    backgroundcolor=\color{backcolour},   
    commentstyle=\color{codegreen},
    keywordstyle=\color{magenta},
    numberstyle=\tiny\color{codegray},
    stringstyle=\color{codepurple},
    basicstyle=\ttfamily\footnotesize,
    breakatwhitespace=false,         
    breaklines=true,                 
    captionpos=b,                    
    keepspaces=true,                 
    numbers=left,                    
    numbersep=5pt,                  
    showspaces=false,                
    showstringspaces=false,
    showtabs=false,                  
    tabsize=2
}

\lstset{style=mystyle}

\title{Introducción a C++}
\subtitle{Estructuras de control}
\author{Tomás Peiretti}
\date{}

\begin{document}

\maketitle

\begin{frame}{Estructuras de control}
    C++ provee estructuras y sentencias de control de flujo que permiten especificar qué es lo que deseamos que nuestro programa haga, como repetir porciones de código o tomar decisiones.
    \begin{columns}
        \column{0.6\textwidth}\begin{itemize}
            \item Estructuras condicionales:
            \begin{itemize}
                \item if-else
                \item switch
            \end{itemize} 
            \item Estructuras iterativas:
            \begin{itemize}
                \item while
                \item do-while
                \item for
            \end{itemize}
            \item Sentencias de salto:
            \begin{itemize}
                \item break
                \item continue
                \item goto
            \end{itemize}
        \end{itemize}
        \column{0.4\textwidth}\includegraphics[width=\textwidth]{cool.png}
    \end{columns}
\end{frame}

\begin{frame}[fragile]{Estructuras condicionales: if-else}
    \centering{if (\alert{expresión condicional})}
\begin{lstlisting}[basicstyle=\tiny]
int main() {
    int x; cin >> x;
    // imprimir el contenido de la variable X solo si x es mayor a 100
    if (x > 100)
        cout << x << endl;
    return 0;
}
\end{lstlisting}
\begin{lstlisting}[basicstyle=\tiny]
int main() {
    int x; cin >> x;
    // imprimir el contenido de la variable X solo si x es mayor a 100
    // sino, imprimir el mensaje "x es menor que 100"
    if (x > 100)
        cout << x << endl;
    else
        cout << "x es menor que 100" << endl;
    return 0;
}
\end{lstlisting}
\begin{lstlisting}[basicstyle=\tiny]
int main() {
    int x; cin >> x;
    // imprimir el contenido de la variable X solo si x es mayor a 100
    if (x > 100) {
        cout << x << endl;
    }
    return 0;
}
    \end{lstlisting}
\end{frame}

\begin{frame}[fragile]{Estructuras condicionales: if-else}
\begin{lstlisting}[basicstyle=\tiny]
int main() {
    char x; cin >> x;
    // si x es una letra minuscula, imprimir X y luego cambiarla por la letra Z
    if (x >= 'a' && x <= 'z') {
        cout << x << endl;
        x = 'Z';
    }
    return 0;
}
\end{lstlisting}
\begin{lstlisting}[basicstyle=\tiny]
int main() {
    int x; cin >> x;
    // si x >= 100, imprimir: "x es mayor a 100"
    // si x > 0 y x <= 100, imprimir: "x se encuentra en (0, 100]"
    // si x <= 0, imprimir "fuera de rango"
    if (x <= 0)
        cout << "fuera de rango" << endl;
    else if (x <= 100)
        cout << " x se encuentra en (0,100]" << endl;
    else
        cout << " x es mayor a 100" << endl;
    return 0;
}
\end{lstlisting}
\end{frame}

\begin{frame}[fragile]{Estructuras condicionales: switch}
\begin{lstlisting}[basicstyle=\tiny]
int main() {
    cout << "seleccione una opcion: a, b o c" << endl;
    char opcion; cin >> opcion;

    switch(opcion) {
    case 'a': 
        cout << ":)" << endl;
        break;
    case 'b':
        cout << ":O" << endl;
        break;
    case 'c':
        cout << ";-)" << endl;
        break;
    default:
        cout << "opcion seleccionada incorrecta" << endl;
    }

    // que seria equivalente a:
    if (opcion == 'a')
        cout << ":)" << endl;
    else if (opcion == 'b')
        cout << ":O" << endl;
    else if (opcion == 'c')
        cout << ":-)" << endl;
    else
        cout << "opcion seleccionada incorrecta" << endl;
    return 0;
}
\end{lstlisting}
\end{frame}

\begin{frame}{Ejercicios de estructuras condicionales}
    \begin{multicols}{2}
        \begin{itemize}
            \item \href{https://judge.beecrowd.com/es/problems/view/1035}{Beecrowd 1035}
            \item \href{https://judge.beecrowd.com/es/problems/view/1037}{Beecrowd 1037}
            \item \href{https://judge.beecrowd.com/es/problems/view/1038}{Beecrowd 1038}
            \item \href{https://judge.beecrowd.com/es/problems/view/1040}{Beecrowd 1040}
            \item \href{https://judge.beecrowd.com/es/problems/view/1041}{Beecrowd 1041}
            \item \href{https://judge.beecrowd.com/es/problems/view/1042}{Beecrowd 1042}
            \item \href{https://judge.beecrowd.com/es/problems/view/1043}{Beecrowd 1043}
            \item \href{https://judge.beecrowd.com/es/problems/view/1044}{Beecrowd 1044}
            \item \href{https://judge.beecrowd.com/es/problems/view/1045}{Beecrowd 1045}
            \item \href{https://judge.beecrowd.com/es/problems/view/1048}{Beecrowd 1048}
            \item \href{https://judge.beecrowd.com/es/problems/view/1049}{Beecrowd 1049}
            \item \href{https://judge.beecrowd.com/es/problems/view/1050}{Beecrowd 1050}
            \item \href{https://judge.beecrowd.com/es/problems/view/1051}{Beecrowd 1051}
        \end{itemize}
    \end{multicols}
\end{frame}

\begin{frame}[fragile]{Estructuras iterativas}
    \begin{columns}
        \column{0.7\textwidth}\begin{lstlisting}[basicstyle=\tiny]
int main() {
    int w = 1;
    while (w <= 100){
        cout << "while: iteracion " << w << endl; 
        w++;
    }

    for (int f = 1 ; f <= 100 ; f++){
        cout << "for: iteracion " << f << endl;
    } 

    int dw = 1;
    do {
        cout << "do-while: iteracion " << dw << endl;
        dw++;
    } while(dw <= 100);
}
\end{lstlisting}
        \column{0.35\textwidth}\includegraphics[width=\textwidth]{iterativas_meme.jpg}
    \end{columns}
\end{frame}

\begin{frame}[fragile]{Estructuras iterativas}
\begin{lstlisting}[basicstyle=\tiny]
int main() {
    // ingresar N enteros por teclado
    int n; cin >> n;
    for (int i = 0 ; i < n ; i++){
        int x; cin >> x;
        cout << "entero nro " << i + 1 << "= " << x << endl;
    }

    // ingresar 2 enteros por teclado hasta fin de archivo (EOF)
    int e1, e2;
    while (cin >> e1 >> e2){
        cout << "Entero 1 = " << e1 << endl;
        cout << "Entero 2 = " << e2 << endl;   
    }
}
\end{lstlisting}
\begin{lstlisting}[basicstyle=\tiny]
int main() {
    // Leer N numeros e imprimir el mayor
    int N, mayor = -1000000;
    cin >> N;
    
    while (N--){
        int x; cin >> x;
        if (x > mayor)
            mayor = x;
    }
    
    cout << "El mayor es: " << mayor << endl;
}
\end{lstlisting}
\end{frame}

\begin{frame}{Ejercicios de estructuras iterativas}
    \footnotesize\begin{multicols}{3}
        \begin{itemize}
            \item \href{https://judge.beecrowd.com/es/problems/view/1059}{Beecrowd 1059}
            \item \href{https://judge.beecrowd.com/es/problems/view/1060}{Beecrowd 1060}
            \item \href{https://judge.beecrowd.com/es/problems/view/1064}{Beecrowd 1064}
            \item \href{https://judge.beecrowd.com/es/problems/view/1065}{Beecrowd 1065}
            \item \href{https://judge.beecrowd.com/es/problems/view/1066}{Beecrowd 1066}
            \item \href{https://judge.beecrowd.com/es/problems/view/1067}{Beecrowd 1067}
            \item \href{https://judge.beecrowd.com/es/problems/view/1070}{Beecrowd 1070}
            \item \href{https://judge.beecrowd.com/es/problems/view/1071}{Beecrowd 1071}
            \item \href{https://judge.beecrowd.com/es/problems/view/1072}{Beecrowd 1072}
            \item \href{https://judge.beecrowd.com/es/problems/view/1073}{Beecrowd 1073}
            \item \href{https://judge.beecrowd.com/es/problems/view/1074}{Beecrowd 1074}
            \item \href{https://judge.beecrowd.com/es/problems/view/1075}{Beecrowd 1075}
            \item \href{https://judge.beecrowd.com/es/problems/view/1078}{Beecrowd 1078}
            \item \href{https://judge.beecrowd.com/es/problems/view/1079}{Beecrowd 1079}
            \item \href{https://judge.beecrowd.com/es/problems/view/1080}{Beecrowd 1080}
            \item \href{https://judge.beecrowd.com/es/problems/view/1095}{Beecrowd 1095}
            \item \href{https://judge.beecrowd.com/es/problems/view/1096}{Beecrowd 1096}
            \item \href{https://judge.beecrowd.com/es/problems/view/1097}{Beecrowd 1097}
            \item \href{https://judge.beecrowd.com/es/problems/view/1098}{Beecrowd 1098}
            \item \href{https://judge.beecrowd.com/es/problems/view/1099}{Beecrowd 1099}
            \item \href{https://judge.beecrowd.com/es/problems/view/1101}{Beecrowd 1101}
            \item \href{https://judge.beecrowd.com/es/problems/view/1113}{Beecrowd 1113}
            \item \href{https://judge.beecrowd.com/es/problems/view/1114}{Beecrowd 1114}
            \item \href{https://judge.beecrowd.com/es/problems/view/1115}{Beecrowd 1115}
            \item \href{https://judge.beecrowd.com/es/problems/view/1116}{Beecrowd 1116}
            \item \href{https://judge.beecrowd.com/es/problems/view/1117}{Beecrowd 1117}
            \item \href{https://judge.beecrowd.com/es/problems/view/1133}{Beecrowd 1133}
        \end{itemize}
    \end{multicols}
\end{frame}

\begin{frame}[fragile]{Sentencias de salto}
    \footnotesize Solo utilizaremos la sentencia \alert{break} para evitar el "encadenamiento" en un switch: \\
    \begin{columns}
        \column{0.5\textwidth}\begin{lstlisting}[basicstyle=\tiny]
int main() {
    char c;
    cin >> c;
    // sin break
    switch(c) {
    case 'A': 
        cout << "A de Avion";
    case 'E':
        cout << "E de Ester";
    case 'I':
        cout << "I de Iglesia";
    case 'O':
        cout << "O de Oso";
    case 'U':
        cout << "U de Uranio";
    }
    // si se ingresa "O", se imprimira:
    // O de OsoU de Uranio
}
\end{lstlisting}
        \column{0.5\textwidth}\begin{lstlisting}[basicstyle=\tiny]
int main() {
    char c;
    cin >> c;
    // con break
    switch(c) {
    case 'A': 
        cout << "A de Avion";
        break;
    case 'E':
        cout << "E de Ester";
        break;
    case 'I':
        cout << "I de Iglesia";
        break;
    case 'O':
        cout << "O de Oso";
        break;
    case 'U':
        cout << "U de Uranio";
    }
    // si se ingresa "O", se imprimira:
    // O de Oso
}
\end{lstlisting}    
    \end{columns}
    \begin{center}
        \scriptsize\url{https://cplusplus.com/doc/tutorial/control/#jumps}
    \end{center}
\end{frame}

\end{document}